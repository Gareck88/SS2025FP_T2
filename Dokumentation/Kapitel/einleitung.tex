\chapter{Einleitung}
\authormargin{Fabian Scherer}

Die zunehmende Digitalisierung der Arbeits- und Kommunikationswelt führt zu einem stetig wachsenden Bedarf an effizienten Methoden zur Erfassung und Dokumentation von Informationen. Meetings und Konferenzen erzeugen eine große Menge an Gesprächsinhalten, deren manuelle Protokollierung jedoch zeitaufwendig, fehleranfällig und mitunter unvollständig ist. Moderne Verfahren der automatischen Spracherkennung (Automatic Speech Recognition, ASR) ermöglichen es, diesen Prozess zu automatisieren und somit die Nachverfolgbarkeit und Effizienz deutlich zu verbessern.

Durch den Einsatz künstlicher Intelligenz (KI) ist es inzwischen möglich, gesprochene Sprache in Echtzeit zuverlässig in Text zu überführen. Neuere Deep-Learning-Ansätze haben die Genauigkeit von Spracherkennungssystemen erheblich gesteigert und erlauben auch die Differenzierung verschiedener Sprecher sowie die Anpassung an Akzente oder Gesprächskontexte.

\authormargin{Fabian Scherer}
Das vorliegende Projekt verfolgt das Ziel, ein KI-gestütztes Spracherkennungssystem zu entwickeln, das Meetings und Konferenzen automatisch transkribiert. Die gesprochene Sprache wird dabei in Echtzeit erkannt, in Text konvertiert und in einer Datenbank gespeichert. Die gespeicherten Protokolle sind durchsuchbar und können den Teilnehmern über einen QR-Code zur Verfügung gestellt werden. Zusätzlich ermöglicht die Integration einer lernfähigen Komponente, dass das System seine Erkennungsgenauigkeit im Laufe der Nutzung kontinuierlich verbessert.

Neben der Zeitersparnis bietet das System auch eine erhöhte Barrierefreiheit, da transkribierte Inhalte insbesondere für Personen mit Hörbeeinträchtigungen oder für nachträgliche Analysen zugänglich gemacht werden. Potenzielle Anwendungsfelder reichen von Unternehmen über den Bildungsbereich bis hin zum Journalismus, wo automatisierte Transkriptionen zur besseren Dokumentation und schnelleren Informationsverarbeitung beitragen können.

Damit leistet das Projekt einen Beitrag zur Weiterentwicklung moderner Kommunikationsunterstützungssysteme, indem es KI-basierte Spracherkennung nicht nur zur Transkription, sondern auch zur intelligenten Aufbereitung und Verfügbarkeit von Gesprächsinhalten nutzbar macht.

\section*{Anmerkung}
\authormargin{Mike Wild}
Der Quelltext der Anwendung ist unter dem folgenden Link zu finden: \url{https://github.com/Gareck88/SS2025FP\_T2}