%----------- File: Kapitel/fazit_ausblick.tex -----------

\chapter{Fazit \& Ausblick}
\label{chap:fazit_ausblick}
\authormargin{Pakize Gökkaya}


Im Rahmen dieser Arbeit wurde die Entwicklung eines KI-gestützten Spracherkennungssystems zur automatischen Transkription von Meetings untersucht und prototypisch umgesetzt. Ausgangspunkt war die Motivation, die manuelle Nachbereitung von Besprechungen durch automatisierte Verfahren zu reduzieren und somit sowohl die Effizienz der Dokumentation als auch die Nachvollziehbarkeit von Entscheidungsprozessen zu erhöhen.

Die durchgeführte Implementierung zeigt, dass eine Kombination aus plattformunabhängiger Audioaufnahme, einer robusten Signalvorverarbeitung sowie dem Einsatz moderner neuronaler Netze in der Lage ist, gesprochene Sprache zuverlässig zu erfassen und in schriftliche Form zu übertragen. Besonders hervorzuheben ist hierbei die modulare Architektur, die es ermöglicht, unterschiedliche Betriebssysteme (Windows, Linux und macOS) zu unterstützen und Audioquellen systemnah zu integrieren. Darüber hinaus konnte durch den Einsatz datenbankgestützter Speicherung eine flexible Verwaltung und Analyse der Transkripte gewährleistet werden.

Die Evaluation des Systems verdeutlicht, dass die erzielten Ergebnisse in Bezug auf Erkennungsrate und Stabilität vielversprechend sind. Gleichwohl wurde auch ersichtlich, dass bestimmte Herausforderungen bestehen bleiben. Dazu gehören insbesondere die variierende Audioqualität in realen Meetingsituationen, die Erkennung von Dialekten oder Akzenten sowie die Bewältigung von Hintergrundgeräuschen und Überschneidungen mehrerer Sprecherinnen und Sprecher. Diese Faktoren beeinflussen die Transkriptionsgenauigkeit weiterhin maßgeblich und erfordern zusätzliche Optimierungen.

Ein weiterer wichtiger Befund betrifft die Integration des Systems in bestehende Arbeitsprozesse. Die prototypische Umsetzung verdeutlicht, dass eine nahtlose Einbindung in gängige Kollaborations- und Kommunikationsplattformen, beispielsweise Microsoft Teams oder Zoom, einen entscheidenden Mehrwert für Anwenderinnen und Anwender bieten könnte. Hier liegt ein erhebliches Potenzial, das über die reine Transkription hinausgeht und auch Aspekte wie automatische Zusammenfassungen, semantische Analyse oder die Extraktion von Aufgaben und Verantwortlichkeiten einschließen könnte.

Darüber hinaus zeigt sich, dass das entwickelte System nicht ausschließlich auf den Meetingkontext beschränkt ist, sondern in einer Vielzahl von Anwendungsszenarien eingesetzt werden kann. So lassen sich beispielsweise Diktate in medizinischen, juristischen oder administrativen Umgebungen effizienter gestalten. Ärztinnen und Ärzte könnten etwa während einer Untersuchung Befunde mündlich erfassen, die anschließend automatisch transkribiert und in elektronische Patientenakten überführt werden. Ein vergleichbares Potenzial ergibt sich in Kanzleien oder in der öffentlichen Verwaltung, wo durch Sprachdiktate die Bearbeitung von Akten und Dokumenten erheblich beschleunigt werden könnte.

Darüber hinaus bietet die Technologie auch in bildungsbezogenen Szenarien einen Mehrwert. Lehrkräfte könnten Vorlesungen oder Seminare in Echtzeit transkribieren lassen, sodass Studierende im Nachgang auf strukturierte Mitschriften zurückgreifen können. Auch für Studierende mit Hörbeeinträchtigungen stellt ein solches System eine wertvolle Unterstützung dar, indem es barrierefreie Zugänge zu gesprochener Sprache schafft.

Ein weiteres Anwendungsfeld liegt im Bereich der Medienproduktion. Journalistinnen und Journalisten profitieren von einer automatisierten Verschriftlichung von Interviews, Pressekonferenzen oder Podcasts, wodurch die Nachbearbeitung und Archivierung von Inhalten erheblich erleichtert wird. In kreativen Prozessen wie Drehbuchentwicklung oder Ideensammlungen kann Spracherkennung zudem als Werkzeug genutzt werden, um spontane Gedanken unmittelbar festzuhalten und strukturiert weiterzuverarbeiten. Schließlich gewinnt auch der Einsatz in alltäglichen Kontexten zunehmend an Bedeutung. Sprachassistenten, Smart-Home-Geräte und mobile Applikationen können durch die Integration robuster Transkriptionsmechanismen erheblich erweitert werden. Hierdurch ergeben sich vielfältige Möglichkeiten, Sprache nicht nur als Eingabemedium, sondern auch als zentrales Element der Mensch-Maschine-Interaktion zu etablieren.

Für zukünftige Arbeiten ergeben sich daher mehrere Perspektiven. Zum einen ist eine Weiterentwicklung des zugrundeliegenden KI-Modells denkbar, etwa durch den Einsatz größerer und stärker spezialisierter Sprachmodelle, die besser auf Mehrsprachigkeit, Fachterminologie oder spontane Gesprächssituationen reagieren können. Zum anderen sollte die Integration von Verfahren der Sprechertrennung (Speaker Diarization) und adaptiven Rauschunterdrückung in Betracht gezogen werden, um die Qualität der Transkripte weiter zu verbessern. Auch die Einbeziehung von Datenschutz- und Sicherheitsaspekten gewinnt zunehmend an Bedeutung, da sensible Gesprächsinhalte verarbeitet werden und hier strenge rechtliche Vorgaben einzuhalten sind.

Zusammenfassend lässt sich festhalten, dass die Arbeit gezeigt hat, wie moderne KI-Technologien einen wesentlichen Beitrag zur Automatisierung und Effizienzsteigerung im Meetingkontext leisten können. Die Ergebnisse bilden eine solide Grundlage, auf der zukünftige Entwicklungen aufbauen können. Es bleibt zu erwarten, dass Spracherkennungssysteme in den kommenden Jahren durch weitere technologische Fortschritte, insbesondere im Bereich der generativen KI, noch leistungsfähiger und vielseitiger werden. Somit eröffnet sich langfristig die Möglichkeit, die Dokumentation und Analyse nicht nur von Besprechungen, sondern auch von Diktaten, Lehrveranstaltungen, Medieninhalten und Alltagsinteraktionen vollständig zu automatisieren und diese als integralen Bestandteil digitaler Arbeitsumgebungen zu etablieren.